\documentclass{article}
\usepackage[utf8]{inputenc}
\usepackage[russian]{babel}
\usepackage[a4paper, top=2cm, bottom=3cm, left=3cm, right=3cm, footskip=1.5cm]{geometry}
\usepackage{indentfirst}
\usepackage{mathtools}
\usepackage{amssymb}

\usepackage{tabularx}
\newcolumntype{C}[1]{>{\hsize=#1\hsize%
\centering\arraybackslash}X}
\renewcommand\tabularxcolumn[1]{m{#1}}

\usepackage{graphicx}
\usepackage{subcaption}
\usepackage{wrapfig}
\usepackage[compatibility=false, margin=10pt, font=normal,
            labelsep=endash, labelfont=sc, textfont=bf, 
            aboveskip=5pt, belowskip=5pt]{caption}

\renewcommand{\theequation}{\thesection.%
\arabic{equation}}

\begin{document}

\section{Обзор}
    Сатурн~--- шестая планета от Солнца и вторая по размерам планета в Солнечной системе после Юпитера. Сатурн обладает заметной системой колец, состоящей главным образом из частичек льда, меньшего количества тяжёлых элементов и пыли (Рис.~\ref{fig:saturn}). Сатурн, а также Юпитер, Уран и Нептун, классифицируются как газовые гиганты. Сатурн назван в честь римского бога земледелия.\par
    Происхождение Сатурна объясняют две основные гипотезы. Согласно гипотезе \textit{«контракции»}, схожесть состава Сатурна с Солнцем в том, что у обоих небесных тел имеется большая доля водорода, и, как следствие, малую плотность можно объяснить тем, что в процессе формирования планет на ранних стадиях развития Солнечной системы в газопылевом диске образовались массивные сгущения, давшие начало планетам, то есть Солнце и планеты формировались схожим образом. Тем не менее, эта гипотеза не может объяснить различия состава Сатурна и Солнца.\par
    Гипотеза \textit{«аккреции»} гласит, что процесс образования Сатурна происходил в два этапа. Сначала в течение 200 миллионов лет шёл процесс формирования твёрдых плотных тел, наподобие планет земной группы. Во время этого этапа из области Юпитера и Сатурна диссипировала часть газа, что затем повлияло на различие в химическом составе Сатурна и Солнца. Затем начался второй этап, когда самые крупные тела достигли удвоенной массы Земли. На протяжении нескольких сотен тысяч лет длился процесс аккреции газа на эти тела из первичного протопланетного облака.\par
    \begin{wrapfigure}[11]{r}[0cm]{0.45\textwidth}
        \vspace{-1pc}
        \includegraphics[width = 0.45\textwidth]{saturn.jpg}
        \caption{\centering{Снимок Сатурна со станции Кассини}}
        \label{fig:saturn}
    \end{wrapfigure}
    На орбите Сатурна находилась автоматическая межпланетная станция «Кассини», запущенная в 1997~году и достигшая системы Сатурна в 2004~году. Основными задачами этой миссии, рассчитанной первоначально на 4~года, являлось изучение структуры и динамики колец и спутников, а также изучение динамики атмосферы и магнитосферы Сатурна и детальное изучение крупнейшего спутника планеты~--- Титана. Основная миссия «Кассини» закончилась в 2008~году, когда аппарат совершил 74 витка вокруг планеты. C 2004~года по 2~ноября 2009 года с помощью аппарата были открыты 8 новых спутников. 15~сентября 2017 года станция завершила свою миссию, сгорев в атмосфере планеты.\par
    Сегодня известно, что у всех четырёх газообразных гигантов есть кольца, но у Сатурна они самые заметные (Рис.~\ref{fig:rings}). Кольца расположены под углом приблизительно 28° к плоскости эклиптики. Кольца не являются сплошным твёрдым телом, а состоят из миллиардов мельчайших частиц, находящихся на околопланетной орбите.\par
    \begin{wrapfigure}[14]{l}[0cm]{0.3\textwidth}
        \vspace{-0.5pc}
        \includegraphics[width = 0.3\textwidth]{rings.jpg}
        \caption{\centering{Кольца Сатурна}}
        \label{fig:rings}
    \end{wrapfigure}
    Существует три основных кольца и четвёртое — более тонкое. Все вместе они отражают больше света, чем диск самого Сатурна. Три основных кольца принято обозначать первыми буквами латинского алфавита. Кольца Сатурна очень тонкие. При диаметре около 250~тыс.км их толщина не достигает и километра (хотя существуют на поверхности колец и своеобразные горы). Несмотря на внушительный вид, количество вещества, составляющего кольца, крайне незначительно. Если его собрать в один монолит, его диаметр не превысил бы 100~км. На изображениях, полученных зондами, видно, что на самом деле кольца образованы из тысяч колец, чередующихся со щелями; картина напоминает дорожки грампластинок. Частички, из которых состоят кольца, имеют размер от 1~сантиметра до 10~метров. По составу они на $93\%$ состоят изо льда с незначительными примесями (которые могут включать в себя сополимеры, образующиеся под действием солнечного излучения, и силикаты) и на $7\%$ из углерода.\par
    Верхние слои атмосферы Сатурна состоят на $96,3\%$ из водорода (по объёму) и на $3,25\%$~--- из гелия. Имеются примеси метана, аммиака, фосфина, этана и некоторых других газов. Аммиачные облака в верхней части атмосферы мощнее юпитерианских. Облака нижней части атмосферы состоят из гидросульфида аммония или воды.\par
    12~ноября 2008 года камеры станции «Кассини» получили изображения северного полюса Сатурна в инфракрасном диапазоне. На них исследователи обнаружили полярные сияния (Рис.~\ref{fig:pole}), подобные которым не наблюдались ещё ни разу в Солнечной системе. Также данные сияния наблюдались в ультрафиолетовом и видимом диапазонах. Полярные сияния представляют собой яркие непрерывные кольца овальной формы, окружающие полюс планеты. Кольца располагаются на широте, как правило, в 70-$80^{\circ}$.\par
    Облака на северном полюсе Сатурна образуют гигантский шестиугольник (Рис.~\ref{fig:pole}). Впервые это обнаружено во время пролётов «Вояджера» около Сатурна в 1980-х годах, подобное явление никогда не наблюдалось ни в одном другом месте Солнечной системы. Шестиугольник располагается на широте $78^{\circ}$, и каждая его сторона составляет приблизительно 13 800~км, то есть больше диаметра Земли. Период его вращения~--- 10~часов 39~минут. Этот период совпадает с периодом изменения интенсивности радиоизлучения, который, в свою очередь, принят равным периоду вращения внутренней части Сатурна.\par
    \begin{wrapfigure}[16]{r}[0cm]{0.4\textwidth}
        \vspace{-0.5pc}
        \includegraphics[width = 0.4\textwidth]{pole.jpg}
        \caption{\centering{Полярное сияние над северным полюсом Сатурна c гексагональным образованием}}
        \label{fig:pole}
    \end{wrapfigure}
    Странная структура облаков показана на инфракрасном изображении, полученном обращающимся вокруг Сатурна космическим аппаратом «Кассини» в октябре 2006~года. Изображения показывают, что шестиугольник оставался стабильным все 20~лет после полёта «Вояджера», причём шестиугольная структура облаков сохраняется во время их вращения. Отдельные облака на Земле могут иметь форму шестиугольника, но, в отличие от них, шестиугольник на Сатурне близок к правильному.\par
    Полного объяснения этого явления пока нет, однако учёным удалось провести эксперимент, который довольно точно смоделировал эту атмосферную структуру. 30-литровый баллон с водой поставили на вращающуюся установку, причём внутри были размещены маленькие кольца, вращающиеся быстрее ёмкости. Чем больше была скорость кольца, тем больше форма вихря, который образовывался при совокупном вращении элементов установки, отличалась от круговой. В этом эксперименте был получен, в том числе, и 6-угольный вихрь.\par

\section{Спутники}
    У Сатурна известно 62 естественных спутника с подтверждённой орбитой, 53 из которых имеют собственные названия. 12 из них открыты при помощи космических аппаратов: «Вояджер-1» (1980), «Вояджер-2» (1981), «Кассини» (2004—2007). Большинство спутников имеет небольшие размеры и состоит из каменных пород и льда. Они очень светлые, имеют высокую отражательную способность. Сравнительные характеристики спутников Сатурна представлены в таблице ниже.\par
    Самый большой спутник Сатурна~--- Титан, диаметр которого составляет 5152~км. Это единственный спутник с очень плотной атмосферой (в 1,5~раза плотнее земной). Она состоит из азота с примесью метана. Учёные предполагают, что условия на этом спутнике схожи с теми, которые существовали на нашей планете 4~миллиарда лет назад, когда на Земле только зарождалась жизнь.\par
    
    \begin{flushright}
        \textit{Таблица 1. Сравнительные характеристики спутников Сатурна}
    \end{flushright}
    
    \begin{center}
        \begin{tabularx}{1\textwidth}{|C{0.15}|C{0.15}|C{0.15}|C{0.15}|C{0.2}|C{0.15}|}
            \hline
    	    Название & Большая полуось, \textit{км} & Период обращения, \textit{дни} & Масса, \textit{кг} & Ускорение свободного падения, \textit{м/$c^2$} & Год открытия\\
            \hline
            Мимас & 185539 & 0.942 & $3.75 \cdot 10^{19}$ & 0.064 & 1789\\
            \hline
            Энцелат & 238042 & 1.370 & $1.1 \cdot 10^{20}$ & 0.111 & 1789\\
            \hline
            Тефия & 294672 & 1.890 & $6.2 \cdot 10^{20}$ & 0.145 & 1684\\
            \hline
            Диона & 377415 & 2.740 & $1.1 \cdot 10^{21}$ & 0.231 & 1684\\
            \hline
            Рея & 527068 & 4.518 & $2.3 \cdot 10^{21}$ & 0.264 & 1672\\
            \hline
            Титан & 1221865 & 15.950 & $1.3 \cdot 10^{23}$ & 1.352 & 1655\\
            \hline
            Япет & 3560854 & 79.330 & $2.0 \cdot 10^{21}$ & 0.223 & 1671\\
            \hline
        \end{tabularx}
    \end{center}

\section{Интегрирование}
    Зададим кусочно заданную функцию~\eqref{eq:func}, определенную на всей числовой прямой:
    \begin{equation}
        f(x) = \begin{cases}
        2x-6, & x \geqslant 3;\\
        x^2 - 2x - 3, & 1 \leqslant x \leqslant 3;\\
        \sin{x}, & 0 \leqslant x \leqslant 1;\\
        e^{2x}, & -5 \leqslant x \leqslant 0;\\
        -3, & x \leqslant -5.
        \end{cases}
        \label{eq:func}
    \end{equation}\par
    Производная~\eqref{eq:diff} данной кусочно-заданной функции будет задаваться производными каждой из функций на их промежутках:
    \begin{equation}
        f'(x) = \begin{cases}
        2, & x \geqslant 3;\\
        2x - 2, & 1 \leqslant x \leqslant 3;\\
        \cos{x}, & 0 \leqslant x \leqslant 1;\\
        2e^{2x}, & -5 \leqslant x \leqslant 0;\\
        0, & x \leqslant -5.
        \end{cases}
        \label{eq:diff}
    \end{equation}\par
    Интеграл~\eqref{eq:integ} этой кусочно-заданной функции на отрезке $\left[-4, 2\right]$ задается следующим выражением и равняется:
    \begin{equation}
        \begin{aligned}
            F(x) = \int\limits_{-4}^{2}f(x)\,dx &= \int\limits_{-4}^{0}e^{2x}\,dx + \int\limits_{0}^{1}\sin{x}\,dx + \int\limits_{1}^{2}\left(x^2-2x-3\right)\,dx \\
            &= \frac{1}{2}e^{2x}\Biggr|_{-4}^{0} - \cos{x}\Biggr|_{0}^{1} + \left(\frac{x^3}{3} - x^2 -3x\right)\Biggr|_{1}^{2}\\
            &= \frac{1}{2} - \frac{1}{e^8} - \cos{1} + 1 + \frac{7}{3} - 3 - 3\\
            &= -\frac{25}{6} - \frac{1}{e^8} - \cos{1}.
        \end{aligned}
        \label{eq:integ}
    \end{equation}
    
\section{Затмения}
    \textbf{Затмение}~--- астрономическая ситуация, при которой одно небесное тело заслоняет свет от другого небесного тела. Наиболее известны лунные и солнечные затмения.\par
    \textbf{Лунное затмение}~(Рис.\,\ref{pic:lunar}) наступает, когда Луна входит в конус тени, отбрасываемой Землёй. Когда Луна во время затмения полностью входит в тень Земли, говорят о \textit{полном лунном затмении}, когда частично — о \textit{частном} затмении. Лунное затмение может наблюдаться на всём полушарии Земли, обращённом в этот момент к Луне. Вид затемнённой Луны с любой точки Земли, где она вообще видна, практически одинаков — в этом состоит коренное отличие лунных затмений от солнечных, которые видны лишь на ограниченной территории.\par
    \textbf{Солнечное затмение}~(Рис.\,\ref{pic:solar}) происходит, когда Луна попадает между наблюдателем и Солнцем, и загораживает его. Самое длительное солнечное затмение произошло 15~января 2010 года в Юго-Восточной Азии и длилось более 11~минут. По астрономической классификации, если затмение хотя бы где-то на поверхности Земли может наблюдаться как полное, оно называется \textit{полным}. Если затмение может наблюдаться только как частное, оно и классифицируется как \textit{частное}. Помимо полных и частных солнечных затмений, бывают \textit{кольцеобразные} затмения~(Рис.\,\ref{pic:circ}). Кольцеобразное затмение происходит, когда в момент затмения Луна находится на большем удалении от Земли, чем во время полного затмения, и конус тени проходит над земной поверхностью, не достигая её.\par
    Кроме лунных и солнечных затмений, на небе происходят затмения других небесных тел, а также искусственные солнечные затмения. Например, планеты могут затмевать звёзды, или космические корабли могут затмевать другие планеты. Подобные явления называются покрытиями.
	\begin{figure}[p]
		\centering
		\begin{subfigure}[b]{0.7\textwidth}
			\centering
			\includegraphics[width = \textwidth]{lunar.jpg}
			\caption{Лунное затмение}
			\label{pic:lunar}
		\end{subfigure}
		\begin{subfigure}[b]{0.7\textwidth}
			\centering
			\includegraphics[width = \textwidth]{solar.jpg}
			\caption{Солнечное затмение}
			\label{pic:solar}
		\end{subfigure}
		\begin{subfigure}[b]{0.7\textwidth}
			\centering
			\includegraphics[width = \textwidth]{circular.jpg}
			\caption{Кольцевое солнечное затмение}
			\label{pic:circ}
		\end{subfigure}
		\caption{Фотографии}
	\end{figure}	

\end{document}
