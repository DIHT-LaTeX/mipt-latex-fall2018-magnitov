\documentclass[12pt]{article}

\usepackage[utf8]{inputenc}
\usepackage[english,russian]{babel}
 
\begin{document}

    \title{Домашняя работа №1}
    \author{Магнитов Михаил}
    \date{}
    \maketitle
    
    \begin{flushright}
        \textit{Audi multa,\\loquere pauca}
    \end{flushright}\par
    \vspace{20pt}
    Это мой первый документ в системе компьютерной вёрстки \LaTeX.
    \begin{center}
        \textsf{\Huge{<<Ура!!!>>}}
    \end{center}\par
    А теперь формулы. \textsc{Формула}~--- краткое и точное словесное выражение, определение или же ряд математических величин, выраженный условными знаками.\par
    \vspace*{15pt}
    \hspace{15pt}{\Large{\textbf{Термодинамика}}}\par
    Уравнение Менделеева--Клайперона~--- уравнение состояния идеального газа, имеющее вид $pV = \nu RT$, где $p$~--- давление, $V$~--- объём, занимаемый газом, $T$~--- температура газа, $\nu$~--- количество вещества газа, а $R$~--- универсальная газовая постоянная.\par
    \vspace*{15pt}
    \hspace{15pt}{\Large{\textbf{Геометрия \hfill Планиметрия}}}\par
    Для \textsl{плоского} треугольника со сторонами $a$, $b$, $c$ и уголом $\alpha$, лежащим против стороны $a$, справедливо соотношение
    $$
    a^2 = b^2 + c^2 - 2bc\cos{\alpha},
    $$
    из которого можно выразить косинус угла треугольника:
    $$
    \cos{\alpha} = \frac{b^2 + c^2 - a^2}{2bc}.
    $$\par
    Пусть $p$ --- полупериметр треугольника, тогда путем несложных преобразований можно получить, что
    $$
    \tg{\frac{\alpha}{2}} = \sqrt{\frac{(p - b)(p - c)}{p(p - a)}},
    $$\par
    \vspace*{1cm}
    \begin{flushleft}
        На сегодня, пожалуй, хватит\dots Удачи!
    \end{flushleft}

\end{document}