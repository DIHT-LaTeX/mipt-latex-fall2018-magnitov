\documentclass[11pt]{article}
\usepackage[russian]{babel}
\usepackage[utf8]{inputenc}
\usepackage[a5paper, top=1cm, bottom=1cm, left=1cm, right=1cm, footskip=0cm]{geometry}
\usepackage{amsthm}
\usepackage{amssymb}
\usepackage{mathtools}

\newtheorem{theorem}{Теорема}[section]
\newtheorem{definition}{Определение}
\DeclareMathOperator{\mat}{Mat}

\begin{document}
    \section[]{Центральная предельная теорема}
        \begin{theorem}[Линдеберга]
            Пусть $\{\xi_k\}_{k \geqslant 1}$ --- независимые случайные величины, $\mathsf{E} \xi_k^2 < + \infty$ $\forall k$. 
            Обозначим $m_k = \mathsf{E} \xi_k$, $\sigma_k^2 = \mathsf{D} \xi_k > 0$; $S_n = \sum_{i=0}^{n} \xi_i$; $\mathsf{D}_n^2 = \sum_{k=1}^{n} \sigma_k^2$ и $F_k(x)$ --- функция распределения $\xi_k$. 
            Пусть выполнено условие Линдеберга:
                $$ \forall \varepsilon > 0\quad \frac{1}{\mathsf{D}_n^2} \sum\limits_{k=1}^{n}\,\int\limits_{\left\{x : |x-m_k| > \varepsilon\mathsf{D}_n \right\}}{(x - m_k)^2 \,dF_k(x)} \xrightarrow[n \rightarrow \infty]{} 0. $$
            Тогда $\frac{S_n - \mathsf{E} S_n}{\sqrt{\mathsf{D} S_n}} \longrightarrow{} \mathcal{N}(0,1), n \rightarrow \infty$.
        \end{theorem}

    \section[]{Гауссовские случайные векторы}
        \begin{definition}
            Случайный вектор $\vec \xi \sim \mathcal{N}(m,\Sigma)$ --- гауссовский, если его характеристическая функция $\varphi_{\xi}(\vec t) = \exp{(i(\vec m, \vec t) - \frac{1}{2}(\Sigma \vec t, \vec t))}$, $\vec m \in \mathbb{R}^n$, $\Sigma$~--- симметричная неотрицательно определенная матрица.
        \end{definition}

        \begin{definition}
            Случайный вектор $\vec \xi$ --- гауссовский, если он представляется в следующем виде: $\vec \xi = A \vec \eta + \vec b$, где $\vec b \in \mathbb{R}^n$, $A \in \mat{(m \times n)}$ и $\eta = (\eta_1, \dots, \eta_m)$~--- независимые и распределенные $\mathcal{N}(0,1)$.
        \end{definition}
        
        \begin{definition}
            Случайный вектор $\vec \xi$ --- гауссовский, если $\forall\lambda \in \mathbb{R}^n$ случайная величина $(\vec \lambda, \vec \xi)$ имеет нормальное распределение.
        \end{definition}
        
        \begin{theorem}[Об эквивалентности определений гауссовских векторов]
            Предыдущие три определения эквивалентны.
        \end{theorem}
        
\end{document}

