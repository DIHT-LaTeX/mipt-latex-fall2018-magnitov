\documentclass[12pt]{article}
\usepackage[russian]{babel}
\usepackage[utf8]{inputenc}
\usepackage{amsthm}
\usepackage{eufrak}
\usepackage{wasysym}
\usepackage[a4paper, top=2cm, bottom=3cm, left=1cm, right=1cm]{geometry}

\newcounter{astrocounter}
\renewcommand{\theastrocounter}{\thesection.\arabic{astrocounter}}
\newenvironment{astronomy}[1]{
   \par\vspace{\baselineskip}\noindent\par
   \refstepcounter{astrocounter}
   \underline{\large{\textbf{Задача \theastrocounter. \text{#1}}}}
   \vspace{5pt}\noindent\par
   }{\vspace*{8pt}\noindent}

\begin{document}
    \section{Задачи по астрономии}
        \begin{astronomy}{H II}
            Предположим, что за пределами солнечного круга кривая вращения Галактики плоская, параметр плато $v$ = 240 км/с. Пусть известно, что диск нейтрального водорода простирается до галактоцентрического расстояния $R_{max}$ = 50 кпк. Мы наблюдаем облако нейтрального водорода на галактической долготе $l = 140^{\circ}$.  Оцените минимально возможное значение лучевой скорости этого облака.
        \end{astronomy}
        
        \begin{astronomy}{Бейрут}
            В какой момент по истинному солнечному времени 1 сентября Регул ($\alpha_1 = 10^h9^m, \delta_1 = 11^{\circ}53^{'}$) и Шератан ($\alpha_2 = 11^h15^m, \delta_2 = 15^{\circ}20^{'}$) находятся на одном альмукантрате в Бейруте ($\varphi = 33^{\circ}53^{'}$)?
        \end{astronomy}
    
        \begin{astronomy}{Dark Matters}
            В некотором скоплении галактик содержится 70 спиральных и 30 элиптических галактик. Известно, что абсолютная звездная величина эллиптических галактик равна $-20$, соотношение масса-светимость составляет $15 \mathfrak{M}_{\odot}/L_{\odot}$. У спиральных галактик в данном скоплении максимальная скорость вращения составляет 210 км/с, соотношение масса-светимость~--- $5 \mathfrak{M}_{\odot}/L_{\odot}$.\par
            Оцените долю темной материи внутри скопления, если масса межгалактического газа на порядок превышает массу галактик, а типичные скорости галактик в скоплении составляют 1000 км/с. Размер скопления составляет 7 Мпк. Абсолютная звёздная величина Млечного Пути~---$-20.9$.
        \end{astronomy}
    
        \begin{astronomy}{Антипланеты}
            Лупа и Пупа живут на антипланетах, обращающихся вокруг звезды с массой $M_* \simeq 10 M_{\odot}$ по эллиптической орбите с фокальным параметром $p = 0.3$ а.е. и эксцентриситетом $e = 0.72$. Как и полагается антипланетам, время от времени звезда находится точно между ними; в этот момент $X$ истинная аномалия $\nu$ планеты Пупы составляет $237^{\circ}$.\\
            Однажды кто-то опять все перепутал, и центральная звезда бесследно исчезла в момент $X$, уменьшив модули скоростей планет в 217 раз. Установите, с каким периодом $T$ планеты бедных астрономов будут обращаться в отсутствие звезды. Известно, что планеты относятся к классу горячих Юпитеров с массой $M \simeq M_{\jupiter}$.
        \end{astronomy}
        
        \begin{astronomy}{К Сатурну!}
            Космический корабль запустили с поверхности Земли к Сатурну по наиболее энергетически выгодной траектории. При движении по орбите корабль пролетел мимо астероида-троянца (624) Гектор.\\
            Определите большую полуось и эксцентриситет полученной орбиты, скорость старта с поверхности Земли, а также угол между направлением на Солнце и на Сатурн в момент старта корабля. Орбиты планет считать круговыми. Оцените относительную скорость корабля и астероида в момент сближения
        \end{astronomy}
    
\end{document}

