\documentclass[11pt]{article}
\usepackage[russian]{babel}
\usepackage[utf8]{inputenc}
\usepackage[a5paper, top=1cm, bottom=1cm, left=1cm, right=1cm, footskip=0cm]{geometry}
\usepackage{amsthm}
\usepackage{amssymb}
\usepackage{mathtools}
\usepackage{eufrak}
\usepackage{wasysym}

\newcounter{astrocounter}
\renewcommand{\theastrocounter}{\thesection.\arabic{astrocounter}}
\newenvironment{astronomy}[1]{
   \par\vspace{5pt}\noindent\par
   \refstepcounter{astrocounter}
   \underline{\large{\textbf{Задача \theastrocounter. \text{#1}}}}
   \vspace{5pt}\noindent\par
   }{\vspace*{10pt}\noindent}

\newtheorem{theorem}{Теорема}[section]
\newtheorem{definition}{Определение}
\DeclareMathOperator{\mat}{Mat}

\begin{document}
    \section[]{Центральная предельная теорема}
        \begin{theorem}[Линдеберга]
            Пусть $\{\xi_k\}_{k \geqslant 1}$ --- независимые случайные величины, $\mathsf{E} \xi_k^2 < + \infty$ $\forall k$. 
            Обозначим $m_k = \mathsf{E} \xi_k$, $\sigma_k^2 = \mathsf{D} \xi_k > 0$; $S_n = \sum_{i=0}^{n} \xi_i$; $\mathsf{D}_n^2 = \sum_{k=1}^{n} \sigma_k^2$ и $F_k(x)$ --- функция распределения $\xi_k$. 
            Пусть выполнено условие Линдеберга:
                $$ \forall \varepsilon > 0\quad \frac{1}{\mathsf{D}_n^2} \sum\limits_{k=1}^{n}\,\int\limits_{\left\{x : |x-m_k| > \varepsilon\mathsf{D}_n \right\}}{(x - m_k)^2 \,dF_k(x)} \xrightarrow[n \rightarrow \infty]{} 0. $$
            Тогда $\frac{S_n - \mathsf{E} S_n}{\sqrt{\mathsf{D} S_n}} \longrightarrow{} \mathcal{N}(0,1), n \rightarrow \infty$.
        \end{theorem}

    \section[]{Гауссовские случайные векторы}
        \begin{definition}
            Случайный вектор $\vec {\xi} \sim \mathcal{N}(m,\Sigma)$~--- гауссовский, если его характеристическая функция $\varphi_{\xi}(\vec{t}\,) = \exp{\left(i\left(\vec{m\,}, \vec{t\,}\right) - \frac{1}{2}\left(\Sigma \vec{t}\,, \vec{t}\, \right)\right)}$, $\vec{m} \in \mathbb{R}^n$, $\Sigma$~--- симметричная неотрицательно определенная матрица.
        \end{definition}

        \begin{definition}
            Случайный вектор $\vec {\xi}$ --- гауссовский, если он представляется в следующем виде: $\vec \xi = A \vec \eta + \vec b$, где $\vec b \in \mathbb{R}^n$, $A \in \mat{(m \times n)}$ и $\eta = (\eta_1, \dots, \eta_m)$~--- независимые и распределенные $\mathcal{N}(0,1)$.
        \end{definition}
        
        \begin{definition}
            Случайный вектор $\vec {\xi}$ --- гауссовский, если $\forall\lambda \in \mathbb{R}^n$ случайная величина $(\vec \lambda\,, \vec {\xi}\,)$ имеет нормальное распределение.
        \end{definition}
        
        \begin{theorem}[Об эквивалентности определений гауссовских векторов]
            Предыдущие три определения эквивалентны.
        \end{theorem}
    
    \newpage
    \section{Задачи по астрономии}
        \begin{astronomy}{H II}
            Предположим, что за пределами солнечного круга кривая вращения Галактики плоская, параметр плато $v = 240$~км/с. Пусть известно, что диск нейтрального водорода простирается до галактоцентрического расстояния $R_{\text{max}} = 50$~кпк. Мы наблюдаем облако нейтрального водорода на галактической долготе $l = 140^{\circ}$.  Оцените минимально возможное значение лучевой скорости этого облака.
        \end{astronomy}
        
        \begin{astronomy}{Бейрут}
            В какой момент по истинному солнечному времени 1 сентября Регул ($\alpha_1 = 10^{\text{h}}\,9^{\text{m}}, \delta_1 = 11^{\circ}\,53'$) и Шератан ($\alpha_2 = 11^{\text{h}}\,15^{\text{m}}, \delta_2 = 15^{\circ}\,20'$) находятся на одном альмукантрате в Бейруте ($\varphi = 33^{\circ}53'$)?
        \end{astronomy}
    
        \begin{astronomy}{Dark Matters}
            В некотором скоплении галактик содержится 70 спиральных и 30 элиптических галактик. Известно, что абсолютная звездная величина эллиптических галактик равна $-20$, соотношение масса-светимость составляет $15 \mathfrak{M}_{\odot}/L_{\odot}$. У спиральных галактик в данном скоплении максимальная скорость вращения составляет 210~км/с, соотношение масса-светимость~--- $5 \mathfrak{M}_{\odot}/L_{\odot}$.\par
            Оцените долю темной материи внутри скопления, если масса межгалактического газа на порядок превышает массу галактик, а типичные скорости галактик в скоплении составляют 1000~км/с. Размер скопления составляет 7~Мпк. Абсолютная звёздная величина Млечного Пути~--- $-20.9$.
        \end{astronomy}
    
        \begin{astronomy}{Антипланеты}
            Лупа и Пупа живут на антипланетах, обращающихся вокруг звезды с массой $M_* \simeq 10 M_{\odot}$ по эллиптической орбите с фокальным параметром $p = 0.3$~а.\,е. и эксцентриситетом $e = 0.72$. Как и полагается антипланетам, время от времени звезда находится точно между ними; в этот момент $X$ истинная аномалия $\nu$ планеты Пупы составляет $237^{\circ}$.\par
            Однажды кто-то опять все перепутал, и центральная звезда бесследно исчезла в момент $X$, уменьшив модули скоростей планет в 217 раз. Установите, с каким периодом $T$ планеты бедных астрономов будут обращаться в отсутствие звезды. Известно, что планеты относятся к классу горячих Юпитеров с массой $M \simeq M_{\jupiter}$.
        \end{astronomy}
        
        \begin{astronomy}{К Сатурну!}
            Космический корабль запустили с поверхности Земли к Сатурну по наиболее энергетически выгодной траектории. При движении по орбите корабль пролетел мимо астероида-троянца (624) Гектор.\par
            Определите большую полуось и эксцентриситет полученной орбиты, скорость старта с поверхности Земли, а также угол между направлением на Солнце и на Сатурн в момент старта корабля. Орбиты планет считать круговыми. Оцените относительную скорость корабля и астероида в момент сближения.
        \end{astronomy}
        
    \newpage
    \section{Отзыв}
        \begin{itemize}
            \item[$\checkmark$] Полезные домашки, позволяющие разобраться с материалом и сразу использовать много показанных функций и тонкостей на лекции
            \item[$\times$] Многовато домашки за раз, возможно стоит давать меньшими порциями (касается конкретно этого задания)
            \item[$\times$] Было вообще ничего не понятно про окружение и счетчики, было мало примеров на эту тему. Было бы хорошо показать, как это делать, на лекции (какой-нибудь аналог домашней работы)
        \end{itemize}

\end{document}

