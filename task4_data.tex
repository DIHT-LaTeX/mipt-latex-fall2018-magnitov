\documentclass{article}
\usepackage[utf8]{inputenc}
\usepackage[russian]{babel}
\usepackage[a5paper, top=2cm, bottom=2cm, left=1cm, right=1cm]{geometry}
\usepackage{indentfirst}
\renewcommand{\thesection}{\thechapter\arabic{section}.}

\usepackage{pgfplots}
\usepackage{graphicx}
\usepackage{subcaption}

\usepackage{color}
\definecolor{mygreen}{RGB}{0,128,0}
\definecolor{myred}{RGB}{195,33,33}
\definecolor{mypurple}{RGB}{170,34,255}

\usepackage{listings}
\lstset{
backgroundcolor=\color{white},
basicstyle=\small\ttfamily,
breaklines=true,
frame=single,
keywordstyle=\color{mygreen},
language=Python,
morekeywords={*,as},
numbers=left,
numbersep=10pt,
numberstyle=\scriptsize\sffamily\color{gray},
stringstyle=\color{myred},
tabsize=4,
showstringspaces=false,
literate=
    {*}{{\textcolor{mypurple}{*}}}{1}
    {+}{{\textcolor{mypurple}{+}}}{1},
}

\usepackage{fancyhdr}
\renewcommand{\footrulewidth}{0.2mm}
\renewcommand{\headrulewidth}{0.2mm}
\pagestyle{fancy}
\fancyhf{}
\fancyfoot[C]{Последнее задание по \LaTeX\,\,на курсе от Студсовета ФИВТ}
\fancyhead[R]{\thepage}
\fancyhead[L]{\leftmark}

\begin{document}
    
    \section{Код}
        Для исследования экспериментальной зависимости вида $y = ax+b$ был написан следующий код на языке Python:
        \begin{figure}[h!]
            \centering
            \begin{subfigure}{0.78\textwidth}
                \lstinputlisting{gen.py}
            \end{subfigure}
        \end{figure}

    \newpage
    \section{График}
        МНК по сгенерированным данным для исследования этой линейной зависимости дает следующие результаты: $a = 3.0612$, $b = 2.6882$. Изобразим на графике сгенерированные точки и их линейную аппроксимацию.
        \begin{figure}[h!]
            \centering
            \begin{tikzpicture}
                \begin{axis}[grid=major, ymax=40, ymin=-40, legend pos=south east, legend style={font=\small}, legend entries={Сгенерированные данные, Линейная аппроксимация}, width=0.8\textwidth, height=0.6\textwidth, xtick={-10,-5, ...,10}, ytick={-30, -15, ..., 30}]
                    \addplot+[x=x, y=y, myred, only marks, mark=o, fill opacity=0.1, mark size=1.5] table{data.dat};
                    \addplot+[domain=-10:10, mygreen, line width=1pt, no marks]{3.0612 * x + 2.6882};
                    \node at (axis cs:2,1) [anchor=south west] {$y = 3.0612x + 2.6882$};
                \end{axis}
            \end{tikzpicture}
        \end{figure}
        
\end{document}
